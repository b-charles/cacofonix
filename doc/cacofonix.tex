\documentclass{article}

% USEPACKAGE
\usepackage{color}
\usepackage{graphicx}
\usepackage{hyperref}
\usepackage{listings}
\usepackage{xspace}
\usepackage{dingbat}
\usepackage{graphics}

% COMMANDS
\newcommand\cacofonix{\textsc{Cacofonix}\xspace}
\newcommand\matlab{\textsc{Matlab}\xspace}

\newcommand\note{\lstinline!Note!\xspace}

\newcommand\frerejaques{\emph{Fr\`ere Jacques}\xspace}

\newcommand\noteFile{\texttt{Note.m}\xspace}
\newcommand\cacofonixFile{\texttt{cacofonix.m}\xspace}
\newcommand\noteInitFile{\texttt{NoteInit.m}\xspace}
\newcommand\noteInitFrFile{\texttt{NoteInit\_fr.m}\xspace}

\newcommand\wikipedia{\textsc{Wikipedia}\xspace}
\newcommand\wiktionary{\textsc{Wiktionary}\xspace}
\newcommand\google{\textsc{Google}\xspace}

\newcommand\footurl[1]{\footnote{\url{#1}}\xspace}
\newcommand\exchange[2]{\texttt{#1}\footnote{\url{#2}}\xspace}

\newcommand\sfz{s$\!f\!$z\xspace}

% ME AND MYSELF
\definecolor{mecolor}{rgb}{1,0.25,0.25}
\definecolor{myselfcolor}{rgb}{0.25,0.25,1}
\newenvironment{meenv}{ \par \noindent \makebox[6em][r]{ \textcolor{mecolor}{Me}: " --~}}{~"}
\newenvironment{myselfenv}{ \par \noindent \makebox[6em][r]{ \textcolor{myselfcolor}{Myself}: " --~}}{~"}
\newcommand{ \me }[1]{%
\begin{meenv}%
	#1%
\end{meenv} }
\newcommand{ \myself }[1]{%
\begin{myselfenv}%
	#1%
\end{myselfenv} }

% MATLAB CODE
\definecolor{matlabcolor}{rgb}{0,0.5,0.5}
\lstset{ basicstyle=\color{matlabcolor}\ttfamily, literate={~}{$\sim$}1 {>>}{$\gg$}1, tabsize=4, numbers=left, numberstyle=\tiny\color{black}, stepnumber=3, numbersep=5pt, numberfirstline=false, firstnumber=1 }

\title{\cacofonix user's guide}
\author{by B}
\date{}

\begin{document}

\maketitle

\me{Hi!}
\myself{Hello!}

\paragraph{}

\emph{Welcome. I and my inner voices will try to explain the features of \cacofonix. I realize my English is far to be perfect and I will be grateful of any suggestion to improve this document.}

\tableofcontents

\section{Foreword}

\cacofonix is a \matlab program, helping to write midi files.

\me{Who are the target customers? Who needs help the write midi files? Who thinks about \matlab to create a midi file? Who wakes up the morning and says "Hoho! Today, I will create a midi file. I think the easiest way is to use \matlab and an underground plug in."?}
\myself{Well, a lot of softwares for editing music and creating midi files exist, free or not, and are more attractive, intuitive, flexible and for better results. Nevertheless, \cacofonix was inspired from \exchange{Theme from Super Mario Brother Song}{http://www.mathworks.com/matlabcentral/fileexchange/8442}, a little project found on the \textsc{Matlab Central}, and the first midi files I produced was some themes from Zelda. So, I guess \cacofonix was created for \matlab geeks\dots}
\me{That's cool. One day, they'll rule the world.}

\paragraph{}

To read this document, you need know some music theory and \matlab bases: at least, read a music sheet and, \matlab side, know what is a script, a function, an array, an object\dots Advanced users know what to do with \lstinline!addpath!, but the easiest way to use \cacofonix is to set the \matlab current directory to the directory of codes (\noteFile and \cacofonixFile).

\section{Tutorial}

To see how \cacofonix works, we will try to transcode this sheet (\frerejaques, a famous French nursery melody):\\
\begin{lilypond}
	\relative c'' { \new Staff { \key g \major \time 2/4 g4 a b g b c d2 d8. e16 d8 c b4 g g d g2 } }
\end{lilypond}

\subsection{Dummy code}

To create a playable note, you have to create two \note objects, one for the tonality, one for the duration, and associate them with the \lstinline!:! operator:
\begin{lstlisting}
mynote = Note('G'):Note(1);
\end{lstlisting}
This instruction creates a G quarter note (\lstinline!Note(4)! for a whole note). Per default, a note without tonality is a rest (but it's a good idea to use \lstinline!Note('rest')!), and, well, a note without duration is not played.

To have a G note (or any other note\dots) of an another octave, you have to use the unary operators \lstinline!-! and \lstinline!+!. \\
\begin{lilypond}
	\relative c' { \new Staff { b1 c a' c c'} }
\end{lilypond}
\begin{lstlisting}
sheet = [ ...
	-Note('B'):Note(4) ...
	Note('C'):Note(4) ...
	Note('A'):Note(4) ...
	+Note('C'):Note(4) ...
	++Note('C'):Note(4) ];
\end{lstlisting}

A complete sheet is just an array of \lstinline!Note! objects. So, this is a first (and very dummy) version of \frerejaques:
\begin{lstlisting}
sheet = [ ...
	Note('G'):Note(1) Note('A'):Note(1) ...
	Note('B'):Note(1) Note('G'):Note(1) ...
	Note('B'):Note(1) +Note('C'):Note(1) ...
	+Note('D'):Note(2) ...
	+Note('D'):Note(3/4) +Note('E'):Note(1/4) ...
	+Note('D'):Note(1/2) +Note('C'):Note(1/2) ...
	Note('B'):Note(1) Note('G'):Note(1) ...
	Note('G'):Note(1) Note('D'):Note(1) ...
	Note('G'):Note(2) ];
\end{lstlisting}

The final step, to create the midi files, is to call the \lstinline!cacofonix! function, with setting the name of the file and the tempo:
\begin{lstlisting}
cacofonix( 'FrereJacques', 'Tempo', 100, sheet );
\end{lstlisting}

Theoretically, a file \texttt{FrereJacques.mid} has been created in the current directory and you can open it with your favorite player.

\subsection{Predefinition of some \note objects}

A way to simplify the writing of a sheet is to predefine some \note objects. Two scripts are proposed to initialize \matlab before the creation of sheets: \noteInitFile and \noteInitFrFile. Of course, with a little practrice, you should be able to create your own initialization file. This is an extract of \noteInitFile:
\begin{lstlisting}
R = Note( 'rest' );

Cf = Note( 'Cf' ); C = Note( 'C' ); Cs = Note( 'Cs' );
Df = Note( 'Df' ); D = Note( 'D' ); Ds = Note( 'Ds' );
Ef = Note( 'Ef' ); E = Note( 'E' ); Es = Note( 'Es' );
Ff = Note( 'Ff' ); F = Note( 'F' ); Fs = Note( 'Fs' );
Gf = Note( 'Gf' ); G = Note( 'G' ); Gs = Note( 'Gs' );
Af = Note( 'Af' ); A = Note( 'A' ); As = Note( 'As' );
Bf = Note( 'Bf' ); B = Note( 'B' ); Bs = Note( 'Bs' );

N = Note( 4 );
N2 = Note( 2 );
N4 = Note( 1 );
N8 = Note( 0.5 );
N16 = Note( 0.25 );
N12 = Note( 1/3 );
\end{lstlisting}

In the real \noteInitFile, some other \note objects are instanced. All this objects will be useful, and explained later.

So, \frerejaques becomes:
\begin{lstlisting}
NoteInit
N8dotted = Note( 3/4 );

sheet = [ ...
	G:N4 A:N4 ...
	B:N4 G:N4 ...
	B:N4 +C:N4 ...
	+D:N2 ...
	+D:N8dotted +E:N16 +D:N8 +C:N8 ...
	B:N4 G:N4 ...
	G:N4 D:N4 ...
	G:N2 ];
\end{lstlisting}

\me{Nice. Why each measure is on a line? A whim again?}
\myself{I think it's a good practice to separate each measure on a different line: bigger projects become more readable, specially with the bars and markers functionalities of \note.}

\subsection{Bars}

\cacofonix is able to check some common mistakes, like a measure with the wrong duration, but it needs bars.

A bar object is constructed with \lstinline!Note('bar')!. It's a very useful feature, so a \lstinline!bar! object is defined in \noteInitFile.

With bars, our example becomes:
\begin{lstlisting}
NoteInit
N8dotted = Note( 3/4 );

sheet = [ ...
	G:N4 A:N4 bar ...
	B:N4 G:N4 bar ...
	B:N4 +C:N4 bar ...
	+D:N2 bar ...
	+D:N8dotted +E:N16 +D:N8 +C:N8 bar ...
	B:N4 G:N4 bar ...
	G:N4 D:N4 bar ...
	G:N2 bar ];
\end{lstlisting}

\me{Always adding \lstinline!bar! at each measure, it's a little boring, isn't it?}
\myself{Yes. But something like \exchange{EditorMacro}{http://www.mathworks.com/matlabcentral/fileexchange/24615} (and probably the new \exchange{\matlab editor API}{http://blogs.mathworks.com/desktop/2011/05/16/matlab-editor-api-examples/}) can be helpful.}

\subsection{Dotted notes}
\label{sec:TutoDottedNotes}

In the last version of \frerejaques, a special \note object has been created to represent a dotted eighth note: \lstinline!N8dotted = Note(3/4);!, but the \lstinline!ctranspose! (\lstinline!'!) operator can be used to add the half duration of the note. For example: \\
\begin{lilypond}
	\relative c'' { \new Staff { \time 2/4 g4. g8 g8. g16 g8. g16 g4.. g16 } }
\end{lilypond}

\begin{lstlisting}
NoteInit

sheet = [ ...
	G:N4' G:N8 bar ...
	G:N8' G:N16 G:N8' G:N16 bar ...
	G:N4'' G:N16 bar ];
\end{lstlisting}

\subsection{Ties and slurs}

Ties and slurs are written with the same operator: \lstinline!not! (\lstinline!~!). The operator can be used on one note to merge it with the previous: \lstinline![G:N2 ~G:N8]!. The operator can also be used on an array of notes to play legato: \lstinline!~[C:N4 E:N4 G:N4 +C:N4]!.

\me{There are an another way to create a note with an exotic duration.}
\begin{myselfenv}%
	Yes, but the good way is always use the \lstinline!'! and \lstinline!~! operators.%
\end{myselfenv}
\begin{meenv}%
	And the convenient way is use an array for the duration with the operator \lstinline!:!, like that: \lstinline!G:[N4 N16]!.%
\end{meenv}

\subsection{Chord}

A chord is a note with several tonalities and one duration. There are two ways to create a chord: a good one, the operator \lstinline!plus! (\lstinline!+!), and a bad-but-sometimes-convenient one, using an array with the operator \lstinline!colon! (you know, \lstinline!:!). \\
\begin{lilypond}
	\relative c' { \new Staff { \time 2/4 <c e g c>2 <g' b d g> <c, e g c> <g' b d g>4( <g b d g>8)( <g b d g> ) } }
\end{lilypond}

\begin{lstlisting}
NoteInit
sheet = [ ...
	C+E+G++C:N2 bar ...
	G+B++D++G:N2 bar ...
	[C E G +C]:N2 bar ...
	[G B +D +G]:[N4 N8 N8] bar ];
\end{lstlisting}

\myself{Something to say?}
\me{No.}
\myself{Good.}

\subsection{Polyphonic sections}
\label{sec:TutoPolyphonic}

A polyphonic section is a set of sheets with the same duration and played at the same time. To create a polyphonic sheet, the \emph{monophonic} sheets have to be combined with the operator \lstinline!mrdivide! (\lstinline!/!). \\
\begin{lilypond}
\new GrandStaff <<
	\new Staff { \time 3/4 \relative c'' <<
		{ c2. <c, e g c>2. } \\
		{ c4( e g) s2. }
		>> }
	\new Staff { \clef bass \relative c { c2.( c) } }
>>
\end{lilypond}
\begin{lstlisting}
NoteInit
sheet = [ ( +C:N2' ) / ~[ C:N4 E:N4 G:N4 ] bar ...
	[ C E G +C ]:N2' bar ] / ...
	[ -C:N2' bar ~-C:N2' bar ];
\end{lstlisting}

\begin{meenv}%
It's too bad: there are no polyphonic section in \frerejaques.%
\end{meenv}
\begin{myselfenv}%
Yes, but it's nice in a canon:

\begin{lstlisting}
>> NoteInit
>> sheet = [ ...
	G:N4 A:N4 bar ...
	B:N4 G:N4 bar ...
	B:N4 +C:N4 bar ...
	+D:N2 bar ...
	+D:N8' +E:N16 +D:N8 +C:N8 bar ...
	B:N4 G:N4 bar ...
	G:N4 D:N4 bar ...
	G:N2 bar ];
>> delay = [ N2 bar N2 bar ];
>> cacofonix( 'FrereJacques', 'Tempo', 100, ...
	[ sheet delay ] / [ delay sheet ] );
\end{lstlisting}%
\end{myselfenv}
\me{100\% rock'n'roll.}

\subsection{Display}

A good way to check errors is to display the sheet: the function \lstinline!display! has been overloaded to represent the sheet. This is the display of \frerejaques:
\begin{lstlisting}
>> NoteInit
>> sheet = [ ...
	G:N4 A:N4 bar ...
	B:N4 G:N4 bar ...
	B:N4 +C:N4 bar ...
	+D:N2 bar ...
	+D:N8' +E:N16 +D:N8 +C:N8 bar ...
	B:N4 G:N4 bar ...
	G:N4 D:N4 bar ...
	G:N2 bar ];
>> sheet
$1///////////////////////////////////////////////$
$|-----------G-----------|-----------A-----------$
$2///////////////////////////////////////////////$
$|-----------B-----------|-----------G-----------$
$3///////////////////////////////////////////////$
$|-----------B-----------|-----------C-----------$
$4///////////////////////////////////////////////$
$|-----------------------D-----------------------$
$5///////////////////////////////////////////////$
$|--------D--------|--E--|-----D-----|-----C-----$
$6///////////////////////////////////////////////$
$|-----------B-----------|-----------G-----------$
$7///////////////////////////////////////////////$
$|-----------G-----------|-----------D-----------$
$8///////////////////////////////////////////////$
$|-----------------------G-----------------------$
\end{lstlisting}

New lines are set according to bars, and a lot of stuff can be displayed: chords, polyphonic sheet, meters, markers, dynamics, tempo\dots Only note's octave is never displayed.

To set the number of characters per quater note, you have to use the class function \lstinline!setNbCharByQuater! (for example: \lstinline!Note.setNbCharByQuater(48)! --- the default value is 24).

\subsection{A little more about the \lstinline!cacofonix! function}

As you know, the \lstinline!cacofonix! function writes the final file. It is possible to give several sheets, generally one sheet by instrument. To do that, you have to precise the wanted instrument before the sheet. Suppose you have two sheets, \lstinline!sheet_violin! and \lstinline!sheet_trumpet!:
\begin{lstlisting}
cacofonix( 'myFile.mid', 'Tempo', 100, ...
	'Violin', sheet_violin, ...
	'Trumpet', sheet_trumpet )
\end{lstlisting}

To know the list of available instruments, type \lstinline!cacofonix -instruments!. You can't play more than 15 sheets (if you reach the limit, may be the polyphonic operator \lstinline!/! can be helpful).

Some percussions are available too, type \lstinline!cacofonix -percussions! to know the list. However, percussion sheets can't have tonality: instead of using tonality notes (\lstinline!Note('G')!), you have to use the void note: \lstinline!V=Note('void')!. There are no restriction about the number of percussion sheet.

\subsection{And now?}

Well, we have a nice version of \frerejaques:
\begin{lstlisting}
NoteInit
sheet = [ ...
	G:N4 A:N4 bar ...
	B:N4 G:N4 bar ...
	B:N4 +C:N4 bar ...
	+D:N2 bar ...
	+D:N8' +E:N16 +D:N8 +C:N8 bar ...
	B:N4 G:N4 bar ...
	G:N4 D:N4 bar ...
	G:N2 bar ];
cacofonix( 'FrereJacques', 'Tempo', 100, sheet );
\end{lstlisting}

\me{Hurray!}

We have seen in this tutorial only very basic features of \cacofonix, but it's probably enough for start.

The script \texttt{ForElise.m} is a good demonstration of the possibilities of \cacofonix: you can see different attacks, markers, dynamics, tempo, sustain\dots The next section will explain more precisely all the features of \cacofonix.

Thank you to have read this document, and for your interest about \cacofonix.

\myself{Enjoy!}

\section{Reference guide}

\subsection{Introduction}

In this reference guide, we will see every possibilities to transcribe any sheet the most quickly and faithfully as far as it can be. We suggest to have read the tutorial before start the next sections. This first introductive section will be about important and general key concepts of \cacofonix.

\me{I feel it will be \dots \textsc{BOOOOOOOORING}!}
\myself{Well, your analysis is short and simplistic, but correct. The point is: if you want to play \emph{Three Blind Mice}, you have everything you need in the tutorial. If you want to play the Schubert's \emph{Trout Quintet}, you should read carefully this guide.}

\subsubsection{Instruments, sheets and channels}

For \cacofonix, the transcription of a sheet is realized with an array of \lstinline!Note! objects. In all this document, we call a sheet a coherent array of \lstinline!Note! for \cacofonix.

\cacofonix doesn't allow to modify the instrument in a sheet: \emph{one} sheet is written for \emph{one} instrument. This rule is still true for a drum set: each element of the drum set (bass drum, toms, hi-hat, cymbals\dots) and more generally each percussion have to has it's own sheet, with void notes (see \ref{sec:VoidNote}) instead of tonality notes.

The polyphony feature modify subtly the rule: \emph{one} instrument can have \emph{several} sheets, merged with the \lstinline!mrdivide! (\lstinline!/!) operator. So, the sheets for the first and the second trumpet can be merged and given to \cacofonix with the instrument \lstinline!'Trumpet'!. This kind of manipulation is extremely useful and powerful, but not completely without dangers\footnote{Mainly because of channel dynamics. See the next point and the section \ref{sec:ChannelDynamics}.}.

A channel is a midi concept: each channel has, among other things, an instrument timbre, a volume and a list of events (start/end to play that note). It's a kind of out of one instrument, but all percussions are regrouped in one channel. \cacofonix links each instrument/sheet couple to a channel -- but, of course, percussions, which are merged in the correct channel. Midi specification allows only $16$ channel: $15$ for regular instruments and one for percussions. \cacofonix doesn't merged automatically sheets with the same instrument: if you want to reduced the number of instrument/sheet couples to fit $15$ channels, you have to merged sheets manually with the \lstinline!/! operator.

\subsubsection{The scope}
\label{sec:scope}

For each features of \cacofonix, there is four possible scopes: note, monophonic, channel or global.
\begin{itemize}
	\item A feature with a note scope will modify only one note (or sometimes, an array of notes), and the next notes are not influenced.
	\item A feature with a monophonic scope is \dots a little more complicated. This kind of features modify the next notes of the monophonic part\footnote{It's a sad misuse of language: a polyphonic section is said composed of \emph{monophonic} parts, even if a monophonic part can contains polyphonic sections itself.}. If the feature is placed in a monophonic section, the other monophonic parts (which are played in parallel) are not modified, influenced or perturbed, in any way. If, in the monophonic part and after the modification, a new polyphonic section starts, the feature will be apply in every new monophonic parts. Last point: after the polyphonic section, the last modification of the last written monophonic part will be active. This is a good example (see section \ref{sec:PlainDynamics} for explications about dynamics):
\begin{lstlisting}
NoteInit
sheet = [ C:N bar ...      % default dynamic: mf
	DynF D:N bar ...       % new dynamic: f
	[ ...                  % start a monophonic section
		E:N bar ...        % current dynamic: f
		DynMP F:N bar ...  % new dynamic: mp
		G:N bar ... 
		A:N bar ...
	] ...                  % end of the monophonic section
	/ ...                  % polyphonic operator
	[ ...                  % new monophonic section
		+E:N bar ...       % current dynamic: f
		+F:N bar ...       % current dynamic: f (not modified)
		DynPP +G:N bar ... % new dynamic: pp
		+A:N bar ...
	] ...                  % end of the polyphonic section
	+C:N bar ...           % current dynamic: pp
	];
\end{lstlisting}
	\item A feature with a channel scope modify at the same time all played monophonic sections of the sheet.
	\item Finally, a feature with a global scope modify all sheets given to the \lstinline!cacofonix! function. For more readability, this features can be grouped in a main sheet, see the section \ref{sec:MainSheet}.
\end{itemize}

\me{I not sure to understand\dots}
\myself{A feature with a note scope modify a note. When it's a monophonic scope, that modify the monophonic section. A channel scope modify the channel, and a global feature modify all sheets. It's almost logical.}
\me{Yes, but the monophonic scope seems tricky.}
\myself{Don't be stressed out. There are more difficult to understand latter.}

\subsubsection{Overview of the reference guide}

After this explanations, the next sections of this guide should can be read in any order. 

To search of a particular point, three tables can help: all features concerning a special use of the \lstinline!Note! constructor are in table \ref{tab:NoteAPI}, member functions and operators are listed in the table \ref{tab:NoteFunctions} and the \lstinline!cacofonix! function API is on the table \ref{tab:CacofonixAPI}.

\begin{table}[p]
	\centering
	\begin{tabular}{llcl}
		Contructor syntax & Description & Scope & Section \\
		\hline
		\lstinline!Note( tonality )! & tonality & N & \ref{sec:Tonalities} \\
		\lstinline!Note( duration )! & duration & N & \ref{sec:Duration} \\
		\lstinline!Note( 'bar' )! & bars & M & \ref{sec:MetersAndBars} \\
		\lstinline!Note( [upper lower] )! & meters & G & \ref{sec:MetersAndBars} \\
		\lstinline!Note( '+X'|'-X' )! & octave & M & \ref{sec:OctaveNote} \\
		\lstinline!Note( 'dynamics', dyn, [ d ] )! & dynamics & M & \ref{sec:Dynamics} \\
		\lstinline!Note( 'dynamics*', dyn, [ d ] )! & channel dynamics & C & \ref{sec:ChannelDynamics} \\
		\lstinline!Note( 'cresc', dur, dyn, [ d ] )! & crescendo & M & \ref{sec:Crescendo} \\
		\lstinline!Note( 'cresc*', dur, dyn, [ d ] )! & channel crescendo & C & \ref{sec:ChannelDynamics} \\
		\lstinline!Note( 'tempo', tempo )! & tempo & G & \ref{sec:Tempo} \\
		\lstinline!Note( 'accel', dur, tempo )! & accelerando & G & \ref{sec:AccelerandoRallentando} \\
		\lstinline!Note( 'fermata', dur, play )! & fermata & G & \ref{sec:Fermata} \\
		\lstinline!Note( 'sustain', 'on'|'off' )! & sustain & C & \ref{sec:Sustain} \\
		\lstinline!Note( 'marker', mark )! & marker & G & \ref{sec:Markers} \\
		\lstinline!Note( 'marker*', mark )! & go to & M & \ref{sec:Goto} \\
		\lstinline!Note( 'void' )! & void note & N & \ref{sec:VoidNote} \\
	\end{tabular}
	\caption[\lstinline!Note! constructor API]{\lstinline!Note! constructor API. Scope column: N means note, M means monophonic, C means channel and G means global.}
	\label{tab:NoteAPI}
\end{table}

\begin{table}[p]
	\centering
	\begin{tabular}{lllcl}
		Member function & Operator & Description & Scope & Section \\
		\hline
		\lstinline!plus! & \lstinline!+! & chord & N & \ref{sec:Chord} \\
		\lstinline!colon! & \lstinline!:! & set duration & N & \ref{sec:SetDuration} \\
		\lstinline!ctranspose! & \lstinline!'! & dot & N & \ref{sec:DottedNotes} \\
		\lstinline!not! & \lstinline!~! & tie & N & \ref{sec:TiesAndSlurs} \\
		\lstinline!uplus!, \lstinline!uminus! & \lstinline!+!, \lstinline!-! & octave & N & \ref{sec:OctaveOperators} \\
		\lstinline!mrdivide! & \lstinline!/! & polyphonic section & N & \ref{sec:Polyphony} \\
		\lstinline!mpower! & \lstinline!^! & transposition & N & \ref{sec:Transpose} \\
		\lstinline!mtimes! & \lstinline!.*! & riff & N & \ref{sec:RepeatRiff} \\
		\lstinline!or! & \lstinline!|! & extraction & C & \ref{sec:Extraction} \\
		\lstinline!display!& & display & C & \ref{sec:Display} \\
		\lstinline!times! & \lstinline!*! & expand & N & \ref{sec:Extraction} \\
		\lstinline!minus! & \lstinline!-! & index & N & \ref{sec:Index} \\
	\end{tabular}
	\caption[\lstinline!Note! member functions]{\lstinline!Note! member functions. Scope column: N means note, M means monophonic, C means channel and G means global.}
	\label{tab:NoteFunctions}
\end{table}

\begin{table}[p]
	\centering
	\begin{tabular}{llcl}
		\lstinline!cacofonix! function syntax & Description & Scope & Section \\
		\hline
		\lstinline!cacofonix( file, ... )! & output file & & \ref{sec:CacofonixFunction} \\
		\lstinline!cacofonix( ..., instrument, sheet, ... )! & add a sheet & C & \ref{sec:CacofonixFunction} \\
		\lstinline!cacofonix( ..., mainSheet, ... )! & main sheet & G & \ref{sec:MainSheet} \\
		\lstinline!cacofonix( ..., 'Tempo', tempo, ... )! & default tempo & G & \ref{sec:Tempo} \\
		\lstinline!cacofonix( ..., 'Velocity', velocity, ... )! & velocity & G & \ref{sec:Velocity} \\
	\end{tabular}
	\caption[\lstinline!cacofonix! function API]{\lstinline!cacofonix! function API. Scope column: N means note, M means monophonic, C means channel and G means global.}
	\label{tab:CacofonixAPI}
\end{table}

\subsection{Tonalities}
\label{sec:Tonalities}

To create a tonality \lstinline!Note! object, you simply have to call the constructor with a string containing the name of the expected note. The basic names are \lstinline!'rest'! for a rest, \lstinline!'void'! for a void note (see section \ref{sec:VoidNote}), \lstinline!'do'!, \lstinline!'re'!, \lstinline!'mi'!, \lstinline!'fa'!, \lstinline!'sol'!, \lstinline!'la'!, \lstinline!'si'! and \lstinline!'a'!, \lstinline!'b'!, \lstinline!'c'!, \lstinline!'d'!, \lstinline!'e'!, \lstinline!'f'! and \lstinline!'g'!. You can add one or two \lstinline!'b'! (\emph{b\'emol}, flat) or \lstinline!'d'! (\emph{di\`ese}, sharp) after \lstinline!'do'!, \lstinline!'re'!, \lstinline!'mi'!\dots and you can add one or two \lstinline!'f'! or \lstinline!'s'! after \lstinline!'a'!, \lstinline!'b'!, \lstinline!'c'!\dots The constructor is case insensitive.

In \noteInitFile and \noteInitFrFile files, only notes with none or one accident are instanced, but something like \lstinline!Eff = Note( 'Eff' );! is completely valid.

\subsection{Duration}
\label{sec:Duration}

To create a duration \lstinline!Note! object, you have to call the constructor with a double, which will be the duration of the note, in number of beats: \lstinline!Note(4)! for a whole note, \lstinline!Note(1)! for a quarter note.

\noteInitFile and \noteInitFrFile define some useful duration: the whole note to the sixteenth note and the triplet quarter note.

\subsection{Chord \lstinline!+!}
\label{sec:Chord}

In the tutorial, the binary \lstinline!plus! operator (\lstinline!+!) is used to create chords, but this operator is more general. It merge two notes (or two array of notes) in one object: tonalities are recorded as a chord and durations are summed.

\myself{The script can become messy very quickly if this operator is not correctly used.}
\me{Yes, there is a lot of possibilities: \lstinline!C+N2+E+N8+G!, or \lstinline![C N2]+[E N8]+G!, or \lstinline!C+[N2 E+[N8 G]]!\dots It's awesome.}
\myself{Every time you don't use that operator correctly and neatly, you make Jesus cry, God kills a kitten, and you will burn in hell to expiate your felony.}

The final note is tied, detached or arpeggiated if in the two arrays, at least on note is tied, detached or arpeggiated. See the section \ref{sec:TiesAndSlurs} for more informations about ties and detached notes and the section \ref{sec:SetDuration} for arpeggios.

\me{Arpeggios? Arpeggi?}
\myself{\wikipedia gives both\footurl{http://en.wikipedia.org/wiki/Arpeggio}, \wiktionary gives only arpeggios\footurl{http://en.wiktionary.org/wiki/arpeggio}. \google finds 3 820 000 results for argeggios and only 634 000 for arpeggi. The \textsc{Oxford Dictionaries} and The \textsc{Cambridge Dictionaries Online} suggest only arpeggios\footurl{http://oxforddictionaries.com/definition/arpeggio}\footurl{http://dictionary.cambridge.org/dictionary/british/arpeggio}. Arpeggios win.}

\subsection{Octave}
\label{sec:Octave}

There are two ways to add or subtract an octave from a note: the unary operators \lstinline!+! and \lstinline!-! and a special note.

\subsubsection{Octave (\lstinline!+! and \lstinline!-! operators)}
\label{sec:OctaveOperators}

The simplest way to modify the octave of a note is the unary operators \lstinline!+! and \lstinline!-!. Some examples: \\
\begin{lilypond}
\relative c' { \new Staff { \time 4/4 g2 c e g a b c c' <e,,, g c c''>1 } }
\end{lilypond}

\begin{lstlisting}
NoteInit
sheet = [ ...
	-G:N2 C:N2 bar ...
	E:N2 G:N2 bar ...
	A:N2 B:N2 bar ...
	+C:N2 ++C:N2 bar ...
	-E+-G+C+++C:N bar ];
\end{lstlisting}

\begin{meenv}%
The operators \lstinline!+! and \lstinline!-! can be used on array. So, instead of:
\begin{lstlisting}
sheet = [ +C:N bar +E:N bar +G:N bar ];
\end{lstlisting}
we can write:
\begin{lstlisting}
sheet = +[ C:N bar E:N bar G:N bar ];
\end{lstlisting}%
\end{meenv}
\myself{Yes, but maybe the special note for the octave is more convenient.}

\subsubsection{Octave (note)}
\label{sec:OctaveNote}

Sometimes, a sheet or a section of a sheet is naturally high or low pitched. So, instead of add \lstinline!+! or \lstinline!-! behind each note
\myself{\dots or using them on a long sheet like a tramp \dots} \\
\dots you can add a special note which means the default octave. To create a note which raises of $n$ octaves, you have call the constructor \lstinline!Note! with a string \lstinline!'+n'! (\emph{e.g.}, \lstinline!Note('+2')! to raise of two octaves). Use a string \lstinline!'-n'! to lower of $n$ octaves. To restore the default octave, you can use \lstinline!Note('+0')! or \lstinline!Note('-0')!. \\
\begin{lilypond}
\relative c' { \new Staff { \time 4/4 g4 a d f, g a d f, g'' f a g g, f a g } }
\end{lilypond}

\begin{lstlisting}
NoteInit
sheet = [ ...
	          -G:N4 -A:N4  D:N4 -F:N4 bar ...
	Note('-1') G:N4  A:N4 +D:N4  F:N4 bar ...
	Note('+1') G:N4 F:N4 A:N4 G:N4 bar ...
	Note('+0') G:N4 F:N4 A:N4 G:N4 bar ];
\end{lstlisting}

\myself{I guess you would talk about the other way to create this kind of \lstinline!Note!?}
\me{Yeah. Instead of using a \lstinline!'+n'! form, you can create a string repeating \lstinline!'+'! $n$ times, like that: \lstinline!Note('+++')! instead of \lstinline!Note('+3')!. Of course, it's also working with \lstinline!'-'!. It's nice when you want raise or lower only one octave, because you can write only \lstinline!Note('+')! and not \lstinline!Note('+1')!.}
\myself{Yes, you're right. But it didn't work when $n$ means $0$: to restore the default octave, you have absolutely to use \lstinline!Note('+0')! or \lstinline!Note('-0')!.}

\subsection{Set duration \lstinline!:!}
\label{sec:SetDuration}

Officially, the \lstinline!colon! operator (\lstinline!:!) associates a tonality note (or an array of tonality notes) with a duration note (or an array of duration notes):
\begin{lstlisting}
>> mynote = tonality:duration;
\end{lstlisting}

In fact, the \lstinline!:! operator is like the \lstinline!+! operator: it merges two array of notes in the same way. The difference is the optional argument, a string which is composed with the characters of the table \ref{table:charsOfColon}: only one accent can be specify (\lstinline!'>'!, \lstinline!'.'!, \lstinline!'^'! or \lstinline!'-'!), the characters \lstinline!'|'!, for a detached note, and \lstinline!'$'!, for an arpeggiated chord, can be added after or before (for more information about detached notes, see the section \ref{sec:TiesAndSlurs}). If the note id tied (and not detached), the arpeggio is ignored.

\begin{table}
	\center
\begin{tabular}{ccp{7cm}}
\lstinline!'>'! & accent & raise the dynamics of the note \\
\lstinline!'.'! & staccato & the note is played only during the half of its duration \\
\lstinline!'^'! & marcato & like an accent and a staccato \\
\lstinline!'-'! & tenuto & like an accent and the delay between the previous note and this one is risen \\
\lstinline!'|'! & detached & the note is detached, even if it is tied \\
\lstinline!'$'! & arpeggio & the chord is arpeggiated. \\
\end{tabular}
\caption{Characters composing the optional string of the \lstinline!:! operator}
\label{table:charsOfColon}
\end{table}

Some examples: \\
\begin{lilypond}
	\new Staff { \time 4/4 \relative c'' { c4\accent b\staccato a\marcato g\tenuto c,4\( ^( c8) g'8 g2\accent \) } }
\end{lilypond}

\begin{lstlisting}
NoteInit
sheet = [ ...
	+C:'>':N4 B:'.':N4 A:'^':N4 G:'-':N4 bar ...
	~[ C:N4 C:N8 G:N8 G:'|>':N2 ] bar ];
\end{lstlisting}

\subsection{Dotted notes \lstinline!'!}
\label{sec:DottedNotes}

The operator \lstinline!ctranspose! (\lstinline!'!) raises the duration of the note, as a dot: one \lstinline!'! increases the duration of the note by half of its original value, $n$ \lstinline!'! serve to lengthen the note by $1 - \frac{1}{2^n}$ of its original duration. For example: \\
\begin{lilypond}
	\relative c'' { \new Staff { \time 2/4 g4. g8 g8. g16 g8. g16 g4.. g16 } }
\end{lilypond}

\begin{lstlisting}
NoteInit

sheet = [ ...
	G:N4' G:N8 bar ...
	G:N8' G:N16 G:N8' G:N16 bar ...
	G:N4'' G:N16 bar ];
\end{lstlisting}

\me{It's exactly the same example as the section \ref{sec:TutoDottedNotes}!}
\myself{Yes, but it's not the same explanation. And it's a good example.}

\subsection{Ties and slurs \lstinline!\~!}
\label{sec:TiesAndSlurs}

The operator \lstinline!not! (\lstinline!~!) is used to defined ties and slurs. Technically, the note preceding a tied note is played for all its duration, and the notes are merged if they have the same tonality. Sometimes, during a slur, two notes with the same tonality can be juxtaposed: \cacofonix will merge them, unless if the second note precise an attack with the \lstinline!'|'! character (the note is \emph{detached}).

If the operator \lstinline!not! is applied on an array of \note, all \note objects in the array will be tied. In this case, the array can contain not playable \note objects (like bars, meters, crescendo\dots): the operator will ignore them. This is some examples: \\
\begin{lilypond}
	\relative c'' { \new Staff { \time 4/4 g4( g) f( a) g( g) f( a) g\( ^( g) f a \) g\( g f a g1 \) } }
\end{lilypond}

\begin{lstlisting}
NoteInit
sheet = [ ...
	G:N4 ~G:N4 F:N4 ~A:N4 bar ...
	~[ G:N4 G:N4 ] ~[ F:N4 A:N4 ] bar ...
	G:N4 ~G:N4 ~F:N4 ~A:N4 bar ...
	~[ G:N4 G:'|':N4 F:N4 A:N4 bar ...
	G:N ] bar ];
\end{lstlisting}

\begin{meenv}%
The operator \lstinline!~! has an other very specific application: it changes a \emph{marker} to a \emph{go to marker}. See the section \ref{sec:Goto} for more details.%
\end{meenv}

\subsection{Meters and bars}
\label{sec:MetersAndBars}

To defined a meter, you have to instantiate a special \note object by calling the constructor \lstinline!Note! with an array of two double \lstinline!meter = Note([upper lower])!. Without surprise, the double \lstinline!upper! indicates how many beats there are in a measure and the double \lstinline!lower! indicates the note value which represents one beat.

\lstinline!Note('bar')! creates an object representing a bar, and you use it every time you have to separate two measures. You can adding bar in tied arrays, in repeated sections (right or left, see \ref{sec:RepeatRiff}), or in expandable sections (only left, see \ref{sec:Expand}).

When several sheets are given to \cacofonix, the first sheet defined the meters. Next sheets have to precise bars, but not meters (meters are in the global scope, although bars are only in the monophonic scope). If another sheet specify meters, the meters can't be different from the first sheet. With the same spirit, in polyphonic sections, bars have to be added everywhere and, if the meters have to be defined, the first monophonic section contains meters objects and the others don't have to, but if they do, meters have to be always identical.

\me{For project with several sheets, it's maybe a good idea if the first sheet is only composed with meters, and maybe the other things whith a global scope like tempo and markers, do you think?}
\myself{Shh! It's explained at the section \ref{sec:MainSheet}. Let him talk.}

The first measure duration is never checked, to allow anacrusis. If no meter is given, the second measure is the reference and all measure have to have the same duration. Even if the first measure is an anacrusis, the other measures have to always have the same duration, or the meter has to be redefined.

Some examples: \\
\begin{lilypond}
\new GrandStaff <<
\new Staff { \clef treble \relative c'' { \time 2/4 \partial 8 c,8 g'2 \time 4/4 c4 d e f \time 6/8 e8\( f e\) e\( d e\) } }
\new Staff { \clef bass \relative c' { \time 2/4 \partial 8 r8 g2 \time 4/4 c,4 d e f \time 6/8 e8\( f e\) e\( d e\) } }
>>
\end{lilypond}

\begin{lstlisting}
NoteInit
sheet = [ ...
	Note( [2 4] ) ...
	C:N8 bar ...
	G:N2 bar ...
	Note( [4 4] ) ...
	+C:N4 +D:N4 +E:N4 +F:N4 bar ...
	Note( [6 8] ) ...
	~[ +E:N8 +F:N8 +E:N8 ] ~[ +E:N8 +D:N8 +E:N8 ] bar ...
	] / [ ...
	R:N8 bar ...
	-G:N2 bar ...
	-C:N4 -D:N4 -E:N4 -F:N4 bar ...
	Note( [6 8] ) ...
	~[ -E:N8 -F:N8 -E:N8 ] ~[ -E:N8 -D:N8 -E:N8 ] bar ...
	];
\end{lstlisting}

\me{You don't have to use meters and bars.}
\myself{Here we go\dots}
\me{It's true: any midi files \cacofonix can produce, you can make the same results with or without.}
\begin{myselfenv}%
	OK, yes, you're right. Adding bars is boring: you spend time for something which don't appear in the final midi file. But, sometimes, you have thinking about something else, and you have get the wrong duration on several notes. Then, you spend \emph{a lot} of time to find where is the mistake on the sheet, correct the code, re-launch \cacofonix and replay the midi files until the next error. Well, with bars, you can still get the wrong tonality of course, but get the wrong duration makes nicely crash \cacofonix, and the error message gives you what is the problematic measure. Look at that:
\begin{lstlisting}
>> NoteInit
>> cacofonix('dumass', [ Note([4 4]) C:N4 bar E:N4 bar G:N bar ])
??? Error using ==> cacofonix
The measure 2 of the sheet 1 is incorrect.
\end{lstlisting}%
\end{myselfenv}
\me{Humpf.}

\subsection{Dynamics}
\label{sec:Dynamics}

Dynamics are really important: they add all the expressiveness.
\me{\dots as far as midi files can be played with expressiveness.}

\subsubsection{Plain dynamics}
\label{sec:PlainDynamics}

\begin{table}
	\centering
	\begin{tabular}{rll}
		\lstinline!'inf'! & & loudest possible \\
		\lstinline!'fff'! & forteissimo possibile & very \emph{very} loud \\
		\lstinline!'ff'! & fortissimo & very loud \\
		\lstinline!'f'! & forte & loud \\
		\lstinline!'mf'! & mezzo-forte & moderately loud (default) \\
		\lstinline!'mp'! & mezzo-piano & moderately soft \\
		\lstinline!'p'! & piano & soft \\
		\lstinline!'pp'! & pianissimo & very soft \\
		\lstinline!'ppp'! & pianissimo possibile & softest possible \\
		\lstinline!'0'! & & silence \\
	\end{tabular}
	\caption{Strings for dynamics}
	\label{tab:dynamics}
\end{table}

To start a section with a precise dynamic, you just have to add a special note: \lstinline!Note('dynamics',dyn)!, where \lstinline!dyn! is a string found in the table \ref{tab:dynamics}. You can also create a dynamic note with a delay: \lstinline!Note('dynamics',dyn,delay)!. The delay is a double or a duration \lstinline!Note! and the new dynamic will be effective only after this duration.

\myself{Two precisions: this kind of dynamics specify only how notes are attacked. Even with the delay parameter, you can't modify the dynamic of a note which has begun to be played. So, you can't play fun dynamics like \sfz.}
\me{\dots yet. The trick is at the section \ref{sec:ChannelDynamics}.}
\myself{The second point is about polyphony: plain dynamics have a monophonic scope, so every monophonic part can have a different current dynamic. It's very useful: in a sheet for piano, the right hand and the left hand can play differently.}
\me{But don't forget to specify changes everywhere!}

Plain dynamics, from \lstinline!'fff'! to \lstinline!'ppp'! are defined in \lstinline!NoteInit! and \lstinline!NoteInit_fr! scripts, in the variables \lstinline!DynFFF! to \lstinline!DynPPP!.

\subsubsection{Crescendo}
\label{sec:Crescendo}

To play a crescendo, you have to add a note created by \lstinline!Note( keyword, duration, dynamics )! or \lstinline!Note( keyword, duration, dynamics, delay )!:
\begin{itemize}
	\item \lstinline!keyword! can be \lstinline!'crescendo'!, \lstinline!'cresc'!, \lstinline!'decrescendo'! or \lstinline!'decresc'!.
		\me{Fun story: you can use indifferently these key words! \lstinline!'crescendo'! and \lstinline!'cresc'! can be used to create a decrescendo, and \lstinline!'decrescendo'! and \lstinline!'decresc'! can be used for a crescendo.}
	\item \lstinline!duration! can be a double or a duration \lstinline!Note! which indicates the duration of the crescendo (or the decrescendo).
	\item \lstinline!dynamics! is a string from the table \ref{tab:dynamics}: it gives the final dynamic.
	\item \lstinline!delay! is an optional double or a duration \lstinline!Note!. Like for the dynamic \lstinline!Note!, the crescendo starts only after this duration (and ends at \lstinline!delay+duration!).
\end{itemize}

Like plain dynamics, \cacofonix look at the current dynamic at the start of the note, and apply this dynamic for the attack of the note. So you can't play crescendo or decrescendo on one note.

\subsubsection{The velocity}
\label{sec:Velocity}

To each dynamic corresponds a velocity: it's a value between $0$ and $127$ which specify the attack of the note, in the midi file. You can redefine the mapping: when you're calling the \cacofonix function, add two more parameters:
\begin{lstlisting}
cacofonix( filename, ..., 'Velocity', velocities, ... )
\end{lstlisting}
where \lstinline!velocities! is an array of $8$ values which gives the velocity values of dynamics \lstinline!'ppp'! to \lstinline!'fff'!. You can't set the velocity of the dynamic \lstinline!'0'! and \lstinline!'inf'!, which are always $0$ and $127$. Per default, the values are \lstinline![ 16 33 49 64 80 96 112 126 ]!.

\subsubsection{Channel dynamics}
\label{sec:ChannelDynamics}

When you create a \lstinline!Note! object with \lstinline!'dynamics'!, \lstinline!'crescendo'!, \lstinline!'cresc'!, \lstinline!'decrescendo'! or \lstinline!'decresc'!, you specify only the attacks of next notes. But there is an other way to modify the dynamics: when you add \lstinline!*! at the end of these keywords (\emph{e.g.} \lstinline!'cresc*'!), you modify the volume of the instrument (or the volume of the \emph{channel}). It's pretty convenient for modify the dynamic of notes after the attack, but never forget that, in polyphonic sections, all monophonic sections will be influenced, and it's only modify the volume, whereas a plain dynamic (without \lstinline!*!) is more realistic and can modify the volume and the timbre of the instrument.

Plain dynamics and channel dynamics are independent: even if the current plain dynamic is set, the current channel dynamic is not modify. Also, you can boost a plain dynamic with a channel dynamic, but the experience shown that mix the two kinds of dynamics is quickly messy and it's often a bad idea.

\subsubsection{Example}

This is an example using dynamics: \\
\begin{lilypond}
	\relative c'' { \new Staff { \time 4/4 c2\ff g a\> b c2\mp b2 g1\< a1\sfz } }
\end{lilypond}

\begin{lstlisting}
NoteInit
sheet = [ ...
	Note( 'dynamics', 'ff' ) +C:N2 G:N2 bar ...
	Note( 'decresc', N, 'mp' ) A:N2 B:N2 bar ...
	+C:N2 B:N2 bar ...
	Note( 'dynamics', 'mf' ) ...
	Note( 'dynamics*', 'mp' ) ...
	Note( 'cresc*', N, 'mf' ) G:N bar ...
	Note( 'dynamics*', 'ff' ) ...
	Note( 'dynamics*', 'mf', N8 ) G:N bar ];
\end{lstlisting}

Look at the switch plain/channel dynamic: the plain dynamic is reset (line 6) and the channel dynamics takes the old value of the plain (line 7).

\subsection{Tempo}
\label{sec:Tempo}

Two ways to set the tempo:
\begin{itemize}
	\item the tempo is constant: in this case, you can just add two input parameters at the calling of the \cacofonix function:
\begin{lstlisting}
cacofonix( filename, ..., 'Tempo', tempo, ... )
\end{lstlisting}
	\item the tempo is not constant: to set a new tempo, add in the sheet a note created with \lstinline!Note('tempo',tempo)!.
\end{itemize}

The tempo is independent of the meters: the double \lstinline!tempo! is always the number of quarter notes (\lstinline!Note(1)!) per minutes.

All sheets share the same tempo in the same time (tempo has a global scope), so you should set the tempo only in one sheet.

\subsubsection{Accelerando/Rallentando}
\label{sec:AccelerandoRallentando}

You can set a new tempo progressively with \lstinline!Note( keyword, duration, newTempo )!, where \lstinline!keyword! is \lstinline!'accelerando'!, \lstinline!'rallentendo'!, \lstinline!'accel.'!, \lstinline!'rall.'!, \lstinline!'accel'! or \lstinline!'rall'!, \lstinline!duration! is the transition duration (a double or a duration \lstinline!Note!), and \lstinline!newTempo! is the final tempo, reached after the given duration.

\subsubsection{Fermata}
\label{sec:Fermata}

You can indicate a fermata with \lstinline!Note( 'fermata', duration, playedDuration )!: \lstinline!duration! is the duration of the fermata (on the sheet) and \lstinline!playedDuration! is wanted duration of the fermata with the current tempo. \lstinline!duration! and \lstinline!playedDuration! can be doubles or duration \lstinline!Note! objects.

\myself{Generally, the \lstinline!playedDuration! is twice bigger than \lstinline!duration!.}
\me{So, why the \lstinline!playedDuration! is not optional with a default value setted to twice \lstinline!duration!?}
\myself{Because I'm lazy. But it's on the todo list.}

\subsubsection{Example}

This is an example of the tempo features: \\
\begin{lilypond}
	\relative c'' { \new Staff { \time 4/4 c4^\markup{ \italic Vivace } b a g f_\markup{ \italic rall... } e d c a'2 g\fermata } }
\end{lilypond}

\begin{lstlisting}
NoteInit
sheet = [ ...
	Note( 'tempo', 140 ) +C:N4 B:N4 A:N4 G:N4 bar ...
	Note( 'rall', N, 100 ) F:N4 E:N4 D:N4 C:N4 bar ...
	A:N2 Note( 'fermata', N2, N ) G:N2 bar ];
\end{lstlisting}

\subsection{Polyphony \lstinline!/!}
\label{sec:Polyphony}

Polyphonic sections are really useful, in particular in two cases: you can't play more than 15 sheets (midi 1.0 specifications allowed only 16 channels, with one channel for percussions), so when you have several sheets with the same instrument, you can merge theses sheets and get an unique sheets by instrument. Or, more often, your instrument has the possibility to play several notes independantly, like a piano with the right and left hands.

Polyphonic sections are easy to use: just merge two sheets with the \lstinline!mrdivide! (\lstinline!/!) operator. The two sheets are called \emph{monophonic sections}, even if they contain smaller polyphonic sections. You can merge more than two sheets: to play \lstinline!sheet1!, \lstinline!sheet2!, \lstinline!sheet3!\dots at in the same time, just apply the operator to add each sheet: \lstinline!sheet1 / sheet2 / sheet3 /...! Only one rule: every sheet has to have the same total duration.

Monophonic sections can contain everything: bars, crescendo, tempo\dots Just be careful to the scope of the feature, as explained at \ref{sec:scope}.

\me{It's funny.}
\myself{Hm hm. The polyphony is funny. Are you high?}
\begin{meenv}%
Have you ever apply the function \lstinline!length! on a sheet?
\begin{lstlisting}
>> NoteInit
>> sheet1 = [ C:N4 D:N4 E:N4 F:N4 ];
>> length( sheet1 )

ans =

     4

>> sheet2 = [ G:N4 A:N4 B:N4 +C:N4 ];
>> length( sheet2 )

ans =

     4

\end{lstlisting}
Nothing wrong. But:
\begin{lstlisting}
>> length( sheet1 / sheet2 )

ans =

    11

\end{lstlisting}
Where come from the $3$ extra notes?%
\end{meenv}
\myself{Well, \cacofonix use extra special notes to know where starts a polyphonic section. It's not a big deal: this kind a ghost notes are really rare and it's completely transparent for you. Why would you apply \lstinline!length! on a sheet anyway?}
\begin{meenv}%
No reason. I guess I was boring. But, wait\dots If sometimes ghost notes are added, it's pretty hazardous to use indexation like \lstinline!extract = mysheet( 5:25 )!. How to get safely an extract of a sheet?%
\end{meenv}
\myself{It's the first time I heard you speak about safety, but, well, get a look at the section \ref{sec:Extraction}.} 

\subsection{Sustain}
\label{sec:Sustain}

For some instruments (piano, essentially) you can manage (press or release) a sustain pedal. Just add the special note \lstinline!Note('Sustain', 'on')! when you want activate the sustain and \lstinline!Note('Sustain', 'off')! when you don't want sustain anymore.

If a \lstinline!Note('Sustain', 'on')! is added, the \cacofonix function adds automatically a \lstinline!Note('Sustain', 'off')! if no one as been specify before.


\subsection{Transpose \lstinline!\^!}
\label{sec:Transpose}

\subsection{Repeat / Riff \lstinline!.*!}
\label{sec:RepeatRiff}

\subsection{Markers}
\label{sec:Markers}

\subsubsection{Go to a marker}
\label{sec:Goto}

\subsubsection{Extraction \lstinline!|!}
\label{sec:Extraction}

\subsection{Display}
\label{sec:Display}

Someting about the display.

\subsection{The void note}
\label{sec:VoidNote}

\me{Before start, a little mood music.}

\vspace{ 0.25in }

\begin{center}
\begin{lilypond}
\paper { line-width = 106\mm }
\layout { indent = 2.0\cm }
\header{
	title = "Symphony No. 5"
	composer = "Beethoven"
}
\score { <<
\new Staff { 
	\set Staff.instrumentName = \markup \center-column { Violin, \line { Clarinet } }
	\clef treble \key es \major \time 2/4 \relative c'' { r8 g\ff[ g g] es2\fermata r8 f[ f f] d2( d2\fermata) }
}
\new Staff {
	\set Staff.instrumentName = \markup \center-column { Viol, \line { Cello, } \line { \concat{ Bass } } }
	\clef bass \key es \major \partcombine { \relative c' { r8 g[ g g] es2\fermata r8 f[ f f] d2( d2\fermata) } } { \relative c { r8 g[ g g] es2 r8 f[ f f] d2( d2) } }
}
>> }
\end{lilypond}
\end{center}

\vspace{ 0.25in }

\myself{And, as recommended by \emph{The Hitchhiker's Guide to the Galaxy}, \textsc{don't panic}. \rightthumbsup}

\subsubsection{Expand \lstinline!*!}
\label{sec:Expand}

\subsubsection{Index \lstinline!-!}
\label{sec:Index}

\subsection{Function \lstinline!cacofonix!}
\label{sec:CacofonixFunction}

\subsubsection{The Main sheet}
\label{sec:MainSheet}

\subsubsection{Percussions}

\end{document}
